::: Math-252 Practice Exam 1
% 1, 2, 3, 4, 5
Using \(\vec u=\langle-4,6,5\rangle\) and \(\vec v=\langle 2,-3,1\rangle\),
\begin{enumerate}[a.]
  \item Find \(\|\vec u\|\) and \(\|\vec v\|\).
  \item Find \(\vec u\cdot\vec v\).
  \item Find the angle \(\theta\) between \(\vec u\) and \(\vec v\).
  \item Find \(\text{proj}_{\vec v}\vec u\).
  \item Find \(\vec u\times\vec v\).
\end{enumerate}
===
\begin{enumerate}[a.]
  \item
    \(\begin{aligned}[t]
      \|\vec u\|&=\sqrt{77} \\
      \|\vec v\|&=\sqrt{14}
    \end{aligned}\)
  \item \(\vec u\cdot\vec v=-21\)
  \item \(\theta=\arccos\left(\frac{-21}{7\sqrt{22}}\right)\)
\end{enumerate}
---
% 6, 7, 8, 9
Using \(P(-4,1,2)\), \(Q(1,-3,4)\), \(R(-1,0,2)\),
\begin{enumerate}[a.]
  \item Find an equation of the plane passing through the points.
  \item Find parametric equations for the line through P and parallel to \(a=\langle 2,-1,4\rangle\).
  \item Find the distance from the point \((5,-3,2)\) to the plane.
  \item Find the area of the parallelogram determined by \(P\), \(Q\), and \(R\).
\end{enumerate}
===
\begin{enumerate}[a.]
  \item \(2x+6y+7z-12=0\)
  \item \(x=2t-4,\ y=-t+1,\ z=4t+2\)
  \item \(D=\frac{6}{\sqrt{89}}\)
  \item \(A=\sqrt{89}\)
\end{enumerate}
---
% 10
Identify the surface \(x=y^2\).
===
Parabolic cylinder
---
% 11
Identify the surface \(4x^2+4y^2+z^2=4\).
===
ANSWER
---
% 12
Identify the surface \(2x^2-3y^2+6z^2=6\).
===
ANSWER
---
% 13
Identify the surface \(x^2-6y+5z^2=0\).
===
ANSWER
---
% 14, 15, 16, 17
A baseball is thrown from the stands 128 feet above the field at an angle of
30 degrees up from the horizontal with an initial speed of 64 feet per second.
\begin{enumerate}[a.]
  \item Give the position vector for any time \(t\).
  \item When will the ball strike the ground?
  \item How far away will the ball strike the ground?
  \item What is the speed of the ball when it strikes the ground?
\end{enumerate}
===
ANSWER
---
% 18, 19, 20, 21
Using \(\vec r(t)=\langle t\cos t,t\sin t,t^2\rangle\) at \(t=0\),
\begin{enumerate}[a.]
  \item Find \(\vec v\) and \(\vec a\).
  \item Find \(\vec T\) and \(\vec N\).
  \item Find \(K\).
  \item By first finding \(a_{\vec T}\) and \(a_{\vec N}\),\\
    express \(a=a_{\vec T}\vec T+a_{\vec N}\vec N\).
\end{enumerate}
===
\begin{enumerate}[a.]
  \item
    \(\begin{aligned}[t]
      \vec v&=\langle-t\sin t+\cos t,t\cos t+\sin t,2t\rangle \\
      \vec a&=\langle-t\cos t-2\sin t,-t\sin t+2\cos t,2\rangle
    \end{aligned}\)
  \item
    \(\begin{aligned}[t]
      \|\vec v\|&=\frac{}{} \\
      \vec T&= \\
      \vec N&=
    \end{aligned}\)
\end{enumerate}
